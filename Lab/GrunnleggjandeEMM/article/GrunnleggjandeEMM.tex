\documentclass[11pt, a4paper]{article}

\usepackage[T1]{fontenc}
\usepackage[utf8]{inputenc}
\usepackage{amsmath}
\usepackage{amssymb}
\usepackage{amsfonts}
\usepackage{amsthm}
\usepackage{bbm}
\usepackage{graphicx}
\usepackage{verbatim}
\usepackage{caption}
\usepackage{subcaption}
\usepackage{subfig}
\usepackage{float}

\newcommand{\vb}{\mathbf}


\renewcommand{\contentsname}{Innhald}
\renewcommand{\abstractname}{Samandrag}

\begin{document}

\begin{titlepage}

  \title{\normalsize FYS1120 Elektromagnetisme\\
    \vspace{10mm}
    \huge Labøving\\
    \vspace{10mm}
    \normalsize{\bf Grunnleggende elektromagnetisk måleteknikk.}}

  \author{Kristian Tuv, Hilde Solesvik Skeie og Øyvind Sigmundson Schøyen.}

\end{titlepage}

\maketitle

\newpage
  \tableofcontents
\newpage

\section*{Indre resistans i eit voltmeter}
\addcontentsline{toc}{section}{Indre resistans i eit voltmeter}

  \subsection*{Oppgåve 1.1}
  \addcontentsline{toc}{subsection}{Oppgåve 1.1}

  \subsection*{PRELAB-Oppgåve 1}
  \addcontentsline{toc}{subsection}{PRELAB-Oppgåve 1}
    Me nytter uttrykket for ladninga til ein kondensator som vert utlada. Denne finn me i boka og er gjeve ved
    \begin{align*}
      q = Q_0e^{-t/RC}.
    \end{align*}
    No setter me inn uttrykket for ladninga gjeve ved spenning og kapasitans.
    \begin{align*}
      C = \frac{Q}{V_{ab}} \qquad \Rightarrow \qquad Q = CV_{ab}.
    \end{align*}
    Då finner me for $V_0 = 2V_1$ at motstanden er gjeve ved
    \begin{align*}
      &CV_1 = CV_0e^{-t/RC} \qquad \Rightarrow \frac{1}{2}V_0 = V_0e^{-t/RC} \\
      &\Rightarrow \qquad \frac{1}{2} = e^{-t/RC} \qquad \Rightarrow \qquad \ln{\frac{1}{2}} = -\frac{t}{RC} \\
      &\Rightarrow \qquad \ln{1} - \ln{2} = -\frac{t}{RC} \qquad \Rightarrow \qquad \ln{2} = \frac{t}{RC} \\
      &\Rightarrow \qquad RC = \frac{t}{\ln{2}} \qquad \Rightarrow \qquad R = \frac{t}{C\ln{2}} \\
      &\Rightarrow \qquad R = \frac{20\text{ s}}{(1\times10^{-6}\text{ F})\ln{2}} \approx 29\text{ M}\Omega.
    \end{align*}

  \subsection*{Oppgåve 1.2}
  \addcontentsline{toc}{subsection}{Oppgåve 1.2}
    Når me lader opp kondensatoren så skjer dette motstandsfritt (med unntak av leidningar). Ved utladning tek det lengre tid då kondensatoren må gjennom voltmeteret som har ein 
    enorm indre resistans.




\newpage


\section*{Indre resistans i eit amperemeter}
\addcontentsline{toc}{section}{Indre resistans i eit amperemeter}


  \subsection*{Oppgåve 2.1}
  \addcontentsline{toc}{subsection}{Oppgåve 2.1}


  \subsection*{PRELAB-Oppgåve 2}
  \addcontentsline{toc}{subsection}{PRELAB-Oppgåve 2}
    Me nytter det same programmet me nytta i utrekningane våre av Hall-effekten.
    \verbatiminput{\string~/Documents/3.Semester/FYS1120/Lab/Hall-effekt/src/PRELABOppgave1.py}

  
  \subsection*{Oppgåve 2.2}
  \addcontentsline{toc}{subsection}{Oppgåve 2.2}
    


\newpage


\section*{Indre resistans i eit termoelement (Peltier-element)}
\addcontentsline{toc}{section}{Indre resistans i eit termoelement (Peltier-element)}


  \subsection*{Oppgåve 3.1}
  \addcontentsline{toc}{subsection}{Oppgåve 3.1}
    Me kopla Peltier-elementet til voltmeteret og la ei hand over elementet. Me såg då at spenninga steig drastisk som ein følgje av temperaturtilførsel frå handa. Når me snudde 
    side på elementet vert spenninga negativ når me auka temperaturen. Ved kopling til ei straumkjelde merka me at elementet vart varmare.


  \subsection*{Oppgåve 3.2}
  \addcontentsline{toc}{subsection}{Oppgåve 3.2}



  \subsection*{Oppgåve 3.3}
  \addcontentsline{toc}{subsection}{Oppgåve 3.3}





\newpage




\section*{Firepunktsmåling av resistans}
\addcontentsline{toc}{section}{Firepunktsmåling av resistans}


  \subsection*{Oppgåve 4.1}
  \addcontentsline{toc}{subsection}{Oppgåve 4.1}




\newpage




\section*{Magnetfeltet til jordkloten}
\addcontentsline{toc}{section}{Magnetfeltet til jordkloten}


  \subsection*{PRELAB-Oppgåve 3}
  \addcontentsline{toc}{subsection}{PRELAB-Oppgåve 3}
    Me nyttar definisjonen av Faradays lov. Då kan me skrive
    \begin{align*}
      \varepsilon = -N\frac{d\Phi_{B}}{dt} = -\frac{d}{dt}\left( NBA\cos(\phi(t)) \right),
    \end{align*}
    kor $\phi(t) = \omega t$. Me deriverer med hensyn på tid og setter inn $\varepsilon = \varepsilon_0$. Det gjer oss
    \begin{align}
      &\varepsilon_0 = NBA\omega\sin(\omega t).
    \end{align}
    Me ser at forholdet mellom $\omega$ og $t_2 - t_1$ gjer oss vinkelen $\phi(t_2 - t_1)$. Då vil
    \begin{align*}
      \omega \propto \frac{1}{t_2 - t_1}.
    \end{align*}
    Frå likning (1) ser me at $\varepsilon_0$ vil ha sin største verdi når $\omega t = \frac{\pi}{2}$.



  \subsection*{Oppgåve 5.1}
  \addcontentsline{toc}{subsection}{Oppgåve 5.1}

\end{document}
