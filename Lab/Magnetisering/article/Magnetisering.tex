\documentclass[11pt, a4paper]{article}

\usepackage[T1]{fontenc}
\usepackage[utf8]{inputenc}
\usepackage{amsmath}
\usepackage{amssymb}
\usepackage{amsfonts}
\usepackage{amsthm}
\usepackage{bbm}
\usepackage{graphicx}
\usepackage{verbatim}
\usepackage{caption}
\usepackage{subcaption}
\usepackage{subfig}
\usepackage{float}

\newcommand{\vb}{\mathbf}


\renewcommand{\contentsname}{Innhald}
\renewcommand{\abstractname}{Samandrag}

\begin{document}

\begin{titlepage}

  \title{\normalsize FYS1120 Elektromagnetisme\\
    \vspace{10mm}
    \huge Labøving 2\\
    \vspace{10mm}
    \normalsize{\bf Magnetisering.}}

  \author{Kristian Tuv, Hilde Solesvik Skeie og Øyvind Sigmundson Schøyen.}

\end{titlepage}

\maketitle

\newpage
  \tableofcontents
\newpage

\section*{Måling av magnetisk susceptibilitet}
  \addcontentsline{toc}{section}{Måling av magnetisk susceptibilitet}

  \subsection*{PRELAB-Oppgåve 1}
    \addcontentsline{toc}{subsection}{PRELAB-Oppgåve 1}
    Viss $\chi > 0$ er materialet paramagnetisk og for $\chi < 0$ er det diamagnetisk. Då ser me at krafta for eit paramagnetisk materiale vil vere gjeve ved
    \begin{align*}
      F_z = -\frac{1}{2\mu_0}\chi A\left( B_1^2 - B_2^2 \right).
    \end{align*}
    Denne krafta peiker i negativ $z$-retning og vil trekke materialet ut av $B$-feltet nedover.
    For eit diamagnetisk materiale vil krafta vere gjeve ved
    \begin{align*}
      F_z = \frac{1}{2\mu_0}\chi A\left( B_1^2 - B_2^2 \right).
    \end{align*}
    Her vert materialet dytta opp i positiv $z$-retning ut av $B$-feltet.


  \subsection*{Oppgåve 1}
    \addcontentsline{toc}{subsection}{Oppgåve 1}
    For å bestemme susceptibiliteten til Vismut vil me måle ein $\Delta m$. Fyrst finner me massa til vismutstaven uten eit magnetfelt. Etterpå slår me på eit kjent 
    $B$-felt og finner den nye massa. Me former deretter om likninga for $F_z$ slik at me kan finne $\chi$.
    \begin{align*}
      F_z = -\frac{1}{2\mu_0}\chi A (B_1^2 - B_2^2) = \Delta mg \qquad \Rightarrow \qquad \chi = -\frac{2\Delta mg \mu_0}{A(B_1^2 - B_2^2)}.
    \end{align*}
    For $B_2 << B_1$ vert likninga omforma til
    \begin{align*}
      \chi = -\frac{2\Delta mg \ mu_0}{AB^2}.
    \end{align*}
    Programmet $\texttt{Oppgave1.py}$ gjer oss utskrifta 
    \begin{align*}
      \input{\string~/Documents/3.Semester/FYS1120/Lab/Magnetisering/src/Oppgave1.txt}.
    \end{align*}
    Me ser at $\chi < 0$ ergo er vismut ein diamagnet.


\newpage


\section*{Måling av magnetisk fluks}
  \addcontentsline{toc}{section}{Måling av magnetisk fluks}


  \subsection*{Oppgåve 2.1}
    \addcontentsline{toc}{subsection}{Oppgåve 2.1}
    For å rekne ut $k$ omformar me likninga
    \begin{align*}
      kDS = V_0t_0
    \end{align*}
    til
    \begin{align*}
      k = \frac{V_0t_0}{DS}.
    \end{align*}
    Me nyttar no verdiane me les av integratoren og frå voltmeteret til å finne $S$, $D$, $V_0$ og $t_0$. $D$ les me av Damp på integratoren, $S$
    finn me ved å la integratoren stabilisere seg på ein verdi. $t_0$ er tida det tek frå me restartar integratoren til ho stabiliserar seg. $V_0$ les av frå 
    voltmeteret som potensialskjelnaden over integratoren. Då får me frå programmet $\texttt{Oppgave2.1.py}$ utskrifta
      \begin{align*}
        \input{\string~/Documents/3.Semester/FYS1120/Lab/Magnetisering/src/Oppgave2.1.txt}.
      \end{align*}



  \subsection*{Oppgåve 2.2}
    \addcontentsline{toc}{subsection}{Oppgåve 2.2}
    For å måle $B$-feltet kopla me ut spenningskjelda og motstandane og kopla til ein spole i integratoren. Deretter held me spola i ein magnet og restartar integratoren. Idet me 
    restartar integratoren trekk me spola ut av magnetfeltet og skriv ned resultatet for $S$. Me tek gjennomsnittsverdien av dei målte $S$-verdiane. Då gjer likninga
    \begin{align*}
      B = \frac{kDS}{NA}
    \end{align*}
    $B$-feltet. Programmet $\texttt{Oppgave2.2.py}$ gjer oss utskrifta
    \begin{align*}
      \input{\string~/Documents/3.Semester/FYS1120/Lab/Magnetisering/src/Oppgave2.2.txt}.
    \end{align*}

  \subsection*{PRELAB-Oppgåve 2}
    \addcontentsline{toc}{subsection}{PRELAB-Oppgåve 2}
    Me nyttar resultatet
    \begin{align}
      \int_{t_1}^{t_2}V\ dt = k\alpha = kDS = -\int_{\Phi_1}^{\Phi_2} \ d\Phi.
    \end{align}
    Frå Gauss lov for magnetisme har me at
    \begin{align*}
      \Phi_B = N\int\vb{B} \cdot d\vb{A}.
    \end{align*}
    Me set dette inn i likning (1) og får
    \begin{align*}
      &\int_{t_1}^{t_2}V \ dt = kDS = \Phi_B = NBA \\ 
      &\Rightarrow \qquad B = \frac{kDS}{NA}.
    \end{align*}



\newpage


\section*{Måling av magnetisk hysterese}
  \addcontentsline{toc}{section}{Måling av magnetisk hysterese}
  Grunna ikkje-fungerande utstyr vert me tildelt ei måling for $I = 4$ A.


  \subsection*{PRELAB-Oppgåve 3}
    \addcontentsline{toc}{subsection}{PRELAB-Oppgåve 3}
    Me deriverer funksjonen
    \begin{align*}
      B &= B_0 + \mu_0H, \\
      \frac{dB}{dH} = \mu_0.
    \end{align*}
    Stigningstalet vil då vere $\mu_0$. Me finner då $M$ ved
    \begin{align*}
      B = B_0 + \mu_0H = \mu_0\left( H + M \right) \qquad &\Rightarrow \qquad B_0 + \mu_0H = \mu_0H + \mu_0M \\
      &\Rightarrow \qquad M = \frac{B_0}{\mu_0}.
    \end{align*}


  \subsection*{Oppgåve 3.1}
    \addcontentsline{toc}{subsection}{Oppgåve 3.1}
      Me nyttar formlane
      \begin{align*}
        H_{maks} = \frac{NI_{maks}}{2\pi R},
      \end{align*}
      \begin{align*}
        B = \mu_0(H + M)
      \end{align*}
      og
      \begin{align*}
        B_{maks} = \frac{kD|S_1 - S_2|}{2\pi A}.
      \end{align*}
      Desse nyttar me til å finne $M$. Programmet $\texttt{Oppgave3.1.py}$ gjer oss verdiane
      \begin{align*}
        \input{\string~/Documents/3.Semester/FYS1120/Lab/Magnetisering/src/Oppgave3.1.txt}.
      \end{align*}
      For $|I| \geq 2$ vil magnetiseringa vere relativt konstant. Viss $I \in [-2, 2]$ vil $M$ variere mykje.






  \subsection*{Oppgåve 3.2}
    \addcontentsline{toc}{subsection}{Oppgåve 3.2}
      For å finne $B_r$ les me av nye verdiar for $S_1$ og $S_2$. Me set $H = 0$ og får frå programmet $\texttt{Oppgave3.1.py}$ utskrifta
      \begin{align*}
        \input{\string~/Documents/3.Semester/FYS1120/Lab/Magnetisering/src/Oppgave3.2.txt}.
      \end{align*}
      Denne verdien vil ikkje endre seg då $H \propto I$ slik at $I = 0$ for $H = 0$.







\newpage

\section*{Programma}
  \addcontentsline{toc}{section}{Programma}

  \subsection*{Oppgave1.py}
    \addcontentsline{toc}{subsection}{Oppgave1.py}
    \verbatiminput{\string~/Documents/3.Semester/FYS1120/Lab/Magnetisering/src/Oppgave1.py}

  \subsection*{Oppgave2.1.py}
    \addcontentsline{toc}{subsection}{Oppgave2.1.py}
    \verbatiminput{\string~/Documents/3.Semester/FYS1120/Lab/Magnetisering/src/Oppgave2.1.py}


  \subsection*{Oppgave2.2.py}
    \addcontentsline{toc}{subsection}{Oppgave2.2.py}
    \verbatiminput{\string~/Documents/3.Semester/FYS1120/Lab/Magnetisering/src/Oppgave2.2.py}

  \subsection*{Oppgave3.1.py}
    \addcontentsline{toc}{subsection}{Oppgave3.1.py}
    \verbatiminput{\string~/Documents/3.Semester/FYS1120/Lab/Magnetisering/src/Oppgave3.1.py}




\end{document}
