\documentclass[11pt, a4paper]{article}

\usepackage[T1]{fontenc}
\usepackage[utf8]{inputenc}
\usepackage{amsmath}
\usepackage{amssymb}
\usepackage{amsfonts}
\usepackage{amsthm}
\usepackage{bbm}
\usepackage{graphicx}
\usepackage{verbatim}
\usepackage{caption}
\usepackage{subcaption}
\usepackage{subfig}
\usepackage{float}

\newcommand{\vb}{\mathbf}


\renewcommand{\contentsname}{Innhald}
\renewcommand{\abstractname}{Samandrag}

\begin{document}

\begin{titlepage}

  \title{\normalsize FYS1120 Elektromagnetisme\\
    \vspace{10mm}
    \huge Labøving 2\\
    \vspace{10mm}
    \normalsize{\bf Magnetisering.}}

  \author{Kristian Tuv, Hilde Solesvik Skeie og Øyvind Sigmundson Schøyen.}

\end{titlepage}

\maketitle

\newpage
  \tableofcontents
\newpage

\section*{Måling av magnetisk susceptibilitet}
  \addcontentsline{toc}{section}{Måling av magnetisk susceptibilitet}

  \subsection*{PRELAB-Oppgåve 1}
    \addcontentsline{toc}{subsection}{PRELAB-Oppgåve 1}
    Viss $\chi > 0$ er materialet paramagnetisk og for $\chi < 0$ er det diamagnetisk. Då ser me at krafta for eit paramagnetisk materiale vil vere gjeve ved
    \begin{align*}
      F_z = -\frac{1}{2\mu_0}\chi A\left( B_1^2 - B_2^2 \right).
    \end{align*}
    Denne krafta peiker i negativ $z$-retning og vil trekke materialet ut av $B$-feltet nedover.
    For eit diamagnetisk materiale vil krafta vere gjeve ved
    \begin{align*}
      F_z = \frac{1}{2\mu_0}\chi A\left( B_1^2 - B_2^2 \right).
    \end{align*}
    Her vert materialet dytta opp i positiv $z$-retning ut av $B$-feltet.


  \subsection*{Oppgåve 1}
    \addcontentsline{toc}{subsection}{Oppgåve 1}


\newpage


\section*{Måling av magnetisk fluks}
  \addcontentsline{toc}{section}{Måling av magnetisk fluks}


  \subsection*{Oppgåve 2.1}
    \addcontentsline{toc}{subsection}{Oppgåve 2.1}


  \subsection*{Oppgåve 2.2}
    \addcontentsline{toc}{subsection}{Oppgåve 2.2}

  \subsection*{PRELAB-Oppgåve 2}
    \addcontentsline{toc}{subsection}{PRELAB-Oppgåve 2}
    Me nyttar resultatet
    \begin{align}
      \int_{t_1}^{t_2}V\ dt = k\alpha = kDS = -\int_{\Phi_1}^{\Phi_2} \ d\Phi.
    \end{align}
    Frå Gauss lov for magnetisme har me at
    \begin{align*}
      \Phi_B = N\int\vb{B} \cdot d\vb{A}.
    \end{align*}
    Me set dette inn i likning (1) og får
    \begin{align*}
      &\int_{t_1}^{t_2}V \ dt = kDS = \Phi_B = NBA \\ 
      &\Rightarrow \qquad B = \frac{kDS}{NA}.
    \end{align*}



\newpage


\section*{Måling av magnetisk hysterese}
  \addcontentsline{toc}{section}{Måling av magnetisk hysterese}


  \subsection*{PRELAB-Oppgåve 3}
    \addcontentsline{toc}{subsection}{PRELAB-Oppgåve 3}
    Me deriverer funksjonen
    \begin{align*}
      B &= B_0 + \mu_0H, \\
      \frac{dB}{dH} = \mu_0.
    \end{align*}
    Stigningstalet vil då vere $\mu_0$. Me finner då $M$ ved
    \begin{align*}
      B = B_0 + \mu_0H = \mu_0\left( H + M \right) \qquad &\Rightarrow \qquad B_0 + \mu_0H = \mu_0H + \mu_0M \\
      &\Rightarrow \qquad M = \frac{B_0}{\mu_0}.
    \end{align*}


  \subsection*{Oppgåve 3.1}
    \addcontentsline{toc}{subsection}{Oppgåve 3.1}




  \subsection*{Oppgåve 3.2}
    \addcontentsline{toc}{subsection}{Oppgåve 3.2}



\end{document}
